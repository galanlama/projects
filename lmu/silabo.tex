\documentclass[letterpaper,12pt]{article}
%\usepackage[small,compact]{titlesec}

% hook me up some entire page type margins
\usepackage{geometry}
\geometry{verbose,letterpaper,tmargin=1in,bmargin=1in,lmargin=1in,rmargin=1in}
\usepackage[utf8]{inputenc}

\begin{document}
\pagestyle{empty}
\begin{center}
\textbf{\large LMU\\El libro de Génesis\\otoño 2010}\\
Benjamín Galán\\bgalan@fuller.edu
\end{center}

\subsection*{Calendario y horario del curso}
We will meet four times. \\
\\
Septiembre 11\\
Octubre 9\\
Noviembre 13\\
Diciembre 4

\subsection*{Descripción del curso:}
Este curso ofrece un estudio detallado del contenido, mensaje, e importancia del libro de Génesis en la teología del Antiguo y el Nuevo Testamentos. A través de presentaciones, lecturas, conversaciones, tareas, y discusiones, los estudiantes serán capaces de identificar los temas más importantes del libro, así como sus personajes principales, su mensaje teológico, su entorno cultural-social-histórico, su conexión con el Nuevo Testamento y la iglesia de hoy. También aprenderá a utilizar diferentes herramientas literarias, históricas, sociales, y teológicas para interpretar el texto de manera profunda y significativa.

\subsection*{Objetivos:}
Al final del curso, los y las estudiantes:
\begin{enumerate}{}{}
\item Habrán identificado y entendido los temas principales del libro de Génesis.
\item Habrán logrado interpretar textos del libro utilizando literarios, históricos, sociales y teológicos.
\item Habrán comprendido la importancia teológica del libro de Génesis en general y de sus narrativas en particular para la iglesia de hoy en nuestro contexto específico.
\item Habrán preparado un estudio interpretativo de algún texto del Génesis como proyecto final.
\end{enumerate}

\subsection*{Lecturas requeridas:}


\subsection*{Preparación para el curso:}

\subsubsection*{Asistencia}
\begin{enumerate}
\item Solo tendremos \textbf{cuatro} clases para este curso. Lo que significa que la asistencia de cada uno de nosotros es crucial. Faltar a una clase equivale a faltar a un cuarto de la clase. De manera que les ruego hagamos un compromiso muy serio para estar en cada una de las clases.
\item Debido a que solo son \textbf{cuatro} clases, también es muy importante que estemos a tiempo. Vamos a cubrir bastante material, así que debemos aprovechar el tiempo de la mejor manera.
\item Asegúrense de traer una Biblia a cada clase. Sería mejor si tenemos varias traducciones en clase. 
\end{enumerate}

\subsubsection*{Integridad académica}
\begin{enumerate}
\item El concepto es bien sencillo. Si utilizan las ideas de alguien más, o sus palabras (no importa cuántas palabras sean), debemos de darles crédito adecuado. Así, cuando hagan sus tareas y consulten algo en un libro, en el internet, o una revista o periódico, debemos de acreditar la fuente de manera adecuada.
\item Utilizar las palabras o ideas de alguien más sin dar crédito adecuado constituye \textbf{plagio}.
\item Pueden dar la información necesaria en paréntesis o en pies de nota.
\item Si alguien no sabe cómo hacerlo, por favor pregunten.
\item Otra práctica problemática es la de copiar el trabajo de algún compañero de trabajo y entregarlo como propio. También debemos evitar esta práctica.
\end{enumerate}

\subsubsection*{Lecturas y tareas escritas}
\begin{enumerate}
\item ????
\item Gómez-Acebo, Isabel ed. \emph{Relectura del Génesis}. En Clave de  Mujer. Bilbao: Desclée DeBrouwer, 1997.
\end{enumerate}

\subsubsection*{Participación}
Ya nos conocemos bien, así que ya saben que para mi la participación de cada uno de ustedes es \textbf{crucial}. Vamos a aprender juntos a leer un texto vital para nuestra fe. No se trata tan solo de un ejercicio mental. Es, más bien, un ejercicio profundamente espiritual. 


\subsection*{Temas y calendario}


\end{document}
